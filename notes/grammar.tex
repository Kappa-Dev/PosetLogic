\section{Logic on posets}

\begin{definition}[Poset]
  \label{def:poset}
  A poset $s=\{E,\tleq\}$ is a finite set of events $E$ and a partial order on events $\tleq\subseteq E\times E$. The \emph{cover} relation on events $\leq\subseteq E\times E$ is the transitive reflexive reduction of $\tleq$.
  We use the predicates $\text{min}(s)$ and $\text{max}(s)$ to retrieve the minimal and maximal events, respectively.

  A labelled poset $s=\{E,\tleq,\labl\}$ is a set with an additional labelling function $\labl$ defined on events.
\end{definition}

\subsection{Grammar}

Let $R$ a set of labels, $\mathcal{S}$ a set of posets and let $\varepsilon = \cup_{s_i\in S} E_i$ be the set of events in $\mathcal{S}$, where $E_i$ is the set of events in $s_i$.
We denote $A,B$ elements of $R$ and $s$ elements of $\mathcal{S}$.

\begin{align*}
  x ::= & x^e ~|~ x^s & \tag{variables on events and posets} \\
  t^s ::= & x^s ~|~ s ~|~ \text{min}(t^s) ~|~ \text{max}(t^s) ~|~ t^s_1 \cap t^s_2 &\tag{terms on posets} \\
  t^e ::= & x^e ~|~ e & \tag{terms on events}\\
  t ::= & t^s ~|~ t^e & \tag{terms}\\
  \\
  \varphi ::= & \exists x.\varphi(x) ~|~ \forall x.\varphi(x) ~|&\tag{quantifiers}\\
  & \neg \varphi ~|~\varphi_1 \wedge \varphi_2~|&\tag{logical connectors}\\
  & t^e\in t^s ~|~ \labl(t^e) = A ~|~ t^e_1 <_{t^s} t^e_2 ~|~ t^e_1 \tleq_{t^s} t^e_2 ~|~ t^e_1\in t^s_1 \dashv t^e_2\in t^s_2~|~ t^s_1 = t^s_2 ~|~ t^s_1\subseteq t^s_2
  & \tag{predicates}\\
\end{align*}

We define the logical connectors $\vee$, $\implies$ in the usual manner.

\subsection{Interpretation}

We interpret the logic on the domain of events and posets.

%an interpretation is the link between syntax and semantics. The functions and predicates are interpreted as their corresponding operations on events and posets.

The predicates $s_1 = s_2$ and $s_1\subseteq s_2$ are interpreted using isomorphisms and embedings in the category of posets defined below (\autoref{sec:posets}).

The interpretation of $s_1\cap s_2$ is the set of common substructures in $s_1$ and $s_2$. We introduce it formally in \autoref{sec:posets}. The predicate $\dashv$ can be seen as a predicate that relates events in different posets. Our interpretation in the restrainted case of graph rewriting systems (and in Kappa) will be that of inhibition (\autoref{def:ref_neg_infl}), as we will see in the next sections. The remaining of function and predicates used in our logic have a standard interpretation.

A \emph{valuation} for $\varphi$ is a function
$v:\text{fv}(\varphi)\to\mathcal{E}\uplus\mathcal{S}$
from the set of free variables of $\varphi$ to the set of events and posets.
%
The evaluation of $\varphi$ is defined below, where a valuation function is needed for the set of free variables of $\varphi$.
%We use two functions $\enct$ to evaluate terms and $\enc$ to evaluate formulas.
\begin{align*}
  \enc{\forall x^s.\varphi}_{v} &= T \iff\text{ for all }s\in\mathcal{S}, \enc{\varphi(s/x)}_{v} = T\\
  \enc{\exists x^s.\varphi}_{v} &= T \iff\text{ for some }s\in\mathcal{S}, \enc{\varphi(s/x)}_{v} = T\\
  \enc{\neg\varphi}_v &= \neg\enc{\varphi} \\
  \enc{\varphi_1\wedge\varphi_2}_v &= \enc{\varphi_1}_v\wedge\enc{\varphi_2}_v\\
  \enc{t^e\in t^s}_v &= T\iff\enct{t^e}_v\in\enct{t^s}_v\\
  \enc{\labl(t^e) = A}_v &= T\iff\labl(\enct{t^e}_v)=A\\
  \enc{t^e_1 <_{t^s} t^e_2} &= T\iff e_1 <_s e_2\text{ where }e_1 = \enct{t^e_1}_v,e_2 = \enct{t^e_2}_v,s = \enct{t^s}_v\\
  \enc{t^e_1 \tleq_{t^s} t^e_2} &= T\iff e_1 \tleq_s e_2\text{ where }e_1 = \enct{t^e_1}_v,e_2 = \enct{t^e_2}_v,s = \enct{t^s}_v\\
  \enc{t^e_1\in t^s_1 \dashv t^e_2\in t^s_2}_v &= T\iff e_1\in s_1 \dashv e_2\in s_2\text{ where }
  e_1 = \enct{t^e_1}_v,e_2 = \enct{t^e_2}_v,s_1 = \enct{t^s_1}_v,s_2 = \enct{t^s_2}_v\\
  \enc{t^s_1\subseteq t^s_2}_v &= T\iff\enct{t^s_1}_v\emb \enct{t^s_2}_v\\
  \\
  \enct{x}_{v} &= v(x)\\
  \enct{e}_{v} &= e\\
  \enct{s}_{v} &= s\\
  \enct{\text{min}(t^s)}_{v} &= \text{min}(\enct{t^s})\\
  \enct{\text{max}(t^s)}_{v} &= \text{max}(\enct{t^s})\\
%  \enct{t^s_1 \cup t^s_2}_{v} &= \enct{t^s_1}_v \cup \enct{t^s_2}_{v}\\
  \enct{t^s_1 \cap t^s_2}_{v} &= \enct{t^s_1}_v \otimes \enct{t^s_2}_{v}\\
\end{align*}

A formula $\varphi$ is satisfiable if there exists $v$ such that $\enc{\varphi}_v = T$.

The denotation of $\varphi$ is the set of valuations for which $\varphi$ evaluates to true. We give the denotation of a formula in~\autoref{app:denote}.

\subsection{Category of posets}
\label{sec:posets}

\begin{definition}[Morphisms on posets]
  Given two posets $s_1$, $s_2$ a morphism $f:s_1\to s_2$ is a total function on events that preserves labels and preserves the cover relation. We denote $s_1\iso s_2$ whenever there is an isomorphism between $s_1$ and $s_2$.

  A morphisms on posets is \emph{injective}, denoted $s_1\emb s_2$, when it is injective on events.
\end{definition}

\begin{lemma}
  Posets and their morphisms form a category.
\end{lemma}

Let us introduce now what we mean by a common substructure to posets.

\begin{definition}
  Let $s_1$ and $s_2$ be two posets.
  A common substructure of $s_1$ and $s_2$ is a span $s_1\overset{f_i}{\remb} m_i\overset{g_i}{\emb} s_2$ such that $\mathit{image}(f_i) \iso \mathit{image}(g_i)$, that is $\forall e\in m_i$, $\labl(f_i(e)) = \labl(g_i(e))$.
\end{definition}

Instead of returning all common substructures we are interested in the "maximal" ones, in the sense of embeddings.

\begin{definition}
  \begin{itemize}
  \item The \emph{multiproduct} of $s_1$ and $s_2$ is the family of spans $(s_1\overset{f_i}{\remb} m_i\overset{g_i}{\emb} s_2)_i$ where for any other span $s_1\overset{f'}{\remb} m'\overset{g'}{\emb} s_2$ there exists a unique $m_i$ and a unique injective morphism $m'\emb m_i$ such that the diagram commutes:
  \[
  \begin{tikzpicture} %[scale=0.8]
    \node (s1) at (-2,0) {\(s_1\)};
    \node (s2) at (2,0) {\(s_2\)};
    \node (m1) at (0,0) {\(m_i\)};
    \node (m2) at (0,-1) {\(m'\)};
    \draw [right hook->] (m1) -- (s1);
    \draw [left hook->] (m1) -- (s2);
    \draw [right hook->] (m2) -- (s1);
    \draw [left hook->] (m2) -- (s2);
    \draw [left hook->, dotted] (m2) -- (m1);
  \end{tikzpicture}
  \]

  \item The restriction of a poset $m=\{E,<,\labl\}$ to a set of events $D\subseteq E$ is the poset $m|D = \{D,<|_D,\labl|_D\}$ where $<|_D$ and $\labl|_D$ is the restriction of $<$ and $\labl$ respectively to the set $D$.

  \item Define $s_1\otimes s_2$ the family of spans $(s_1\overset{f_i'}{\remb} m_i'\overset{g_i'}{\emb} s_2)_i$ such that for each span in $s_1\otimes s_2$ there is a span $s_1\overset{f_i}{\remb} m_i\overset{g_i}{\emb} s_2$ in the multiproduct of $s_1$ and $s_2$ with
   \[ m_i' = m_i|D_i \]
    for $D_i = \{ e\in m_i : \labl(f_i(e)) = \labl(g_i(e))\}$.
  \end{itemize}
\end{definition}

%% \begin{lemma}
%%   The product of two posets $s_0=\{E_0,<,\labl_0\}$ and $s_1=\{E_1,<,\labl_1\}$, denoted $s_0\times s_1 = \{E,<,\labl\}$, is the poset obtained as follows:
%%   \begin{itemize}
%%   \item $E=E_0\times E_1$ is the product on sets, with $e=(e_0,e_1)\in E$ and the two projections $\pi_0(e_0,e_1) = e_0$ and $\pi_1(e_0,e_1) = e_1$;
%%   \item $\labl:E_0\times E_1 \to \mathcal{R}\times\mathcal{R}$ is defined as $\labl(e) = \big(\labl(\pi_0(e)),\labl(\pi_1(e))\big)$. Let us also introduce the two projections on labels $\lambda_{0,1}:\mathcal{R}\times\mathcal{R}\to\mathcal{R}$, defined as $\lambda_0(A,B)=A$ and $\lambda_1(A,B)=B$.
%%   \item $e< e' \iff \pi_0(e)<\pi_0(e')$ and $\pi_1(e)<\pi_1(e')$
%%   \end{itemize}
%%   The two projections morphisms $(\pi_0,\lambda_0):s_0\times s_1\to s_0$ and $(\pi_1,\lambda_1):s_1\times s_1\to s_1$ are given by the projections on the underlying sets and labels.
%% \end{lemma}

%% \begin{definition}
%%   \begin{itemize}
%%   \item A synchronisation algebra on the set of labels $\mathcal{R}\cup\{0\}$ consists of a binary, commutative and associative operation $\cdot$ such that
%%     \[ A\cdot A = A \text{ and }A\cdot B = 0\text{ when }A\neq B\]
%%     for all $A,B\in \mathcal{R}\cup\{0\}$.
%%   \item The synchronised product of two posets $s_0=\{E_0,<,\labl_0\}$ and $s_1=\{E_1,<,\labl_1\}$, denoted $s_0\otimes s_1 = \{E,<,\labl\}$, is obtained as follows:
%%     \begin{itemize}
%%     \item let $s_{\text{prod}} = s_0\times s_1$ be the product with $(\pi_0,\lambda_0)$ and $(\pi_1,\lambda_1)$ the two projections;
%%     \item define $D\subseteq\mathcal{R}\times\mathcal{R}$ a set of labels such that
%%       \[ D = \{ a\in \mathcal{R}\times\mathcal{R} : \lambda_0(a)\cdot \lambda_1(a) = 0\}; \]
%%     \item remove from $s_{\text{prod}}$ the set of events $E_D = \{e\in E_{\text{prod}} : \labl(e)\in D \}$;
%%     \item define $s_0\otimes s_1 = \{E_{\text{prod}}\setminus E_D,<,r\circ\labl\}$ where $r$ is the following relabelling function
%%       \[ r(a) = \lambda_0(a)\cdot \lambda_1(a). \]
%%     \end{itemize}
%%   \end{itemize}
%% \end{definition}
