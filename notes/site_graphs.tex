\section{Site graphs}

\subsection{The category of site-graphs}

Define $\kn = \set{A, B,..}$ a set of agent types %with a distinct agent type \emph{free},
and equipped with a site map $\site:\kn\to\nat$.
%such that $\site(\text{free}) = 0$.
Define $\pn$ a set of properties.

%Define the set of external links as follows:
%\[
%\text{Ext} = \set{-,\sfree}\cup\Sigma_{ag}\cup\set{(A,i) : A\in\Sigma_{ag}, i\in\site(A)}
%\]

\begin{definition}[Site-graphs]
A site-graph is a structure $(\ag,\type,\nodes,\links,p_k)$ where
\begin{itemize}
\item $\ag$ is a set of agents, ranged over by $a,b$;
\item $\type:\ag\to \set{A, B,...}$ assigns a type to each agent;
\item $\nodes\subseteq \ag\times\nat\uplus\{\text{free}\}$ is a set of nodes, with a special node free, and where each other node is a pair $(a,i)$ with $a\in\ag$ an agent and $i<\site(\type(a))$ a site of $a$;
\item $\links\subseteq \nodes\times\nodes$ is a set of edges that is \emph{conflict free}: $\forall e,e'\in\links$, $e=e'$ or $e\cap e' \subseteq \{\text{free}\}$;
\item $p_k\subseteq\nodes$, with $k\in\pn$, is the set of internal states of a site. The pairwise intersection of $\{p_k\}_{k\in\pn}$ is the empty set, that is a site can only have one internal state.
\end{itemize}
Denote $\varepsilon$ the empty $\Sigma$-graph.
\end{definition}

%% \begin{property}
%%   A site with a specified property need not be a \emph{specified} site of an agent: $p_k\not\subseteq\sites$.
%% \end{property}
%% Define a partial order on $\links\cup\text{Ext}$:
%% \[
%% - \leq A \leq (A,i) \leq (a,i)
%% \]

\begin{definition}[Morphisms]
A morphism $h:G\to H$ is a total function on agents $h:\ag_G\to\ag_H$ such that
\begin{itemize}
\item it preserves agent's types:
\[
\type(a) = \type(h(a)) \text{ for all }a \in \ag_G
\]
\item it preserves nodes:
\[
\text{ if }(a,i)\in \nodes_G \text{ then }(h(a),i) \in \nodes_H\text{ and }h(\text{free})=\text{free}
\]
\item it preserves property sets:
\[
\set{h(u) : u\in p_{k,G}} \subseteq p_{k,H}, \forall k\in\pn.
\]
\end{itemize}
An embedding or a mono, denoted $h:G\lemb H$, is an injective morphism on edges, that is it preserves edges:
\[
(u,v)\in\nodes_G \implies (h(u),h(v))\in\nodes_H
\]
\end{definition}

\begin{lemma}
  Site-graphs and morphisms form a category.
\end{lemma}


\begin{definition}[Rules]
  A rule, or an action map, $\alpha:L \action R$ is a span $L\overset{h}{\remb} D \overset{g}{\lemb} R$ such that $h$ and $g$ are monos on site graphs and the following hold
  \begin{itemize}
  %\item for any span $L\overset{h'}{\remb} D' \overset{g'}{\lemb} R$ and any embedding $D\overset{f}{\lemb}D'$ such that $h=h'f$ and $g=g'f$ then $f$ is an isomorphism;
  \item $\forall a\in\ag_D$, $(a,i)\in\nodes_D\iff (h(a),i)\in\nodes_L \iff (g(a),i)\in\nodes_R$;
  %\item if $[(g(a),i),x] \in\links_R$ with $x\in\text{Ext}\setminus \sfree$, then $[(h(a),i),x] \in\links_D$;
  %\item if $[(a,i),x] \in\links_R$ and $\nexists b$ such that $h(b)=a$ or $\nexists y$ such that $[(b,i),y]\in\links_D$ then $x\in\sites_R$;
  \item if $a\in\ag_R$ and $a\notin\text{image}(g)$ then $(a,i)\in\nodes_R$, $\forall i\in\Sigma_{ag-st}(\type(a))$.
  \end{itemize}
\end{definition}

\begin{example}
For the rule
\begin{verbatim}
  A(x!1), C(y!1), B(y) -> A(x!1), C(y), B(y!1)
\end{verbatim}
the domain of definition is \verb|A(x?), C(y?), B(y?)|.
\end{example}

The category of site graphs and morphisms does not have all the pushouts. %but does have all pullbacks.

\begin{example}[Inexistence of pushout in the category of site graphs]
  Let $G$, $H$ and $O$ three site graphs defined as follows\footnote{Henceforth we write the symmetric reduction for the set $\links$.}:
  \begin{align*}
    \ag_G = \{a\},   \type(a) = A,   \links_G = \{[(a,i),B] \} \\
    \ag_H = \{a'\},  \type(a') = A,  \links_H = \{[(a',i),C]\} \\
    \ag_o = \{a''\}, \type(a'') = A, \links_O = \{[(a'',i),\_]\}.
  \end{align*}
  There is no site graph $M$ such that the following diagram commutes:
  \[
  \begin{tikzpicture} %[scale=0.8]
    \node (o) at (0,-1) {\(O\)};
    \node (l1) at (-1.5,0) {\(G\)};
    \node (l2) at (1.5,0) {\(H\)};
    \node (m) at (0,1) {\(M\)};
    \draw [right hook->] (o) -- (l1);
    \draw [left hook->] (o) -- (l2);
    \draw [right hook->] (l1) -- (m);
    \draw [left hook->] (l2) -- (m);
  \end{tikzpicture}
  \]
\end{example}


\begin{lemma}
  Let $L\overset{h}{\remb} D \overset{g}{\lemb} R$ be a rule and let $M$ and $m:L\emb M$ be a site graph and matching, respectively. The DPO rewriting can be applied whenever the gluing conditions hold.
\end{lemma}
\begin{proof}

\end{proof}
